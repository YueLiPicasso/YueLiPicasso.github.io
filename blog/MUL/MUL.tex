\documentclass{article}
\usepackage{amsthm}
\usepackage{amsmath}
\usepackage {graphicx}
\newtheorem*{proposition}{Proposition}
\newtheorem*{lemma}{Lemma}

\title{On Overflow of Unsigned Integer Multiplication}
\author{Yue LI}
\date{\today}



\begin{document}
\maketitle
The x86 unsigned multiplication instruction MUL multiplies RAX with some quadword and stores the result in RDX:RAX.  The overflow flag OF and carry flag CF are set if the high-order bits of the result, which are stored in RDX, are non-zero. The notion of overflow applicable to the MUL instruction is therefore that multiplication of two $n$-bit (unsigned) integers has a result that is longer than $n$ bits. Taking into account that MUL actually uses $2n$ bits to store the result of multiplication of two $n$-bit numbers, in this blog we ask and answer a different overflow question: \emph{is it possible that multiplying two n-bit unsigned integers and the result is longer than 2n bits?}

We shall answer negatively, and prove the following Proposition.

\begin{proposition}
  For all $n=1,2,3,\ldots$, the result of multiplication of two unsigned n-bit integers has at most $2n$ bits.
\end{proposition}

Since an all-ones, like 11111111111, is the largest $n$-bit unsigned integer for every $n$, if we can show that two $n$-bit all-ones multiply and the result is no more than $2n$ bits, then the Proposition easily follows. 

\begin{lemma}
  For all $n=1,2,3,\ldots$, the square of the largest n-bit unsigned integer has at most $2n$ bits.
\end{lemma}

Our proof for the Lemma is inductive, but before examining the formal arguments, let's warm up by looking at some particular cases.

\section{Pattern Discovery for Induction}

Below we show that the Lemma holds for $1\leq n\leq 6$. We shall use sum of powers of 2 to denote an unsigned binary integer, then e.g. 1111 is $2^3+2^2+2^1+2^0$. Equation~\ref{eq0} shows that $1^2$ has 1 bit.

\begin{equation} \label{eq0}
(2^0)^2=2^0
\end{equation}

 Next, when building Equation~\ref{eq1} we use Equation~\ref{eq0} (as underlined).


 \begin{eqnarray} \label{eq1}
   &  (2^1+2^0)^2 \nonumber\\
   =\ & (2^1)^2+2\cdot2^1\cdot2^0+\underline{(2^0)^2} \nonumber\\
   =\  & (2^2+2^2)+\underline{2^0}_{\mbox{ by (\ref{eq0})}} \nonumber\\
   =\  & 2^3+2^0
\end{eqnarray}

Equation~\ref{eq1} shows that $11^2$ has 4 bits. Next, when building Equation~\ref{eq2} we use Equation~\ref{eq1}.

\begin{eqnarray} \label{eq2}
    & (2^2+2^1+2^0)^2\nonumber\\
   =\ & (2^2)^2+2\cdot2^2(2^1+2^0)+\underline{(2^1+2^0)^2}\nonumber\\
   =\ & 2^4+2^3(2^1+2^0)+\underline{2^3+2^0}_{\mbox{ by (\ref{eq1})}}\nonumber\\
   =\ & (2^4+2^4)+(2^3+2^3)+2^0\nonumber\\
   =\ & 2^5+2^4+2^0
\end{eqnarray}

Equation~\ref{eq2} shows that $111^2$ has 6 bits. Next, when building Equation~\ref{eq3} we use Equation~\ref{eq2}.

\begin{eqnarray} \label{eq3}
    & (2^3+2^2+2^1+2^0)^2\nonumber\\
   =\ & (2^3)^2+2\cdot2^3(2^2+2^1+2^0)+\underline{(2^2+2^1+2^0)^2}\nonumber\\
   =\ & 2^6+2^4(2^2+2^1+2^0)+\underline{2^5+2^4+2^0}_{\mbox{ by (\ref{eq2})}}\nonumber\\
   =\ & (2^6+2^6)+(2^5+2^4)+(2^5+2^4)+2^0\nonumber\\
   =\ & 2^7+2^6+2^5+2^0
\end{eqnarray}

Equation~\ref{eq3} shows that $1111^2$ has 8 bits. Next, when building Equation~\ref{eq4} we use Equation~\ref{eq3}.


\begin{eqnarray} \label{eq4}
    & (2^4+2^3+2^2+2^1+2^0)^2\nonumber\\
   =\ & (2^4)^2+2\cdot2^4(2^3+2^2+2^1+2^0)+\underline{(2^3+2^2+2^1+2^0)^2}\nonumber\\
   =\ & 2^8+2^5(2^3+2^2+2^1+2^0)+\underline{2^7+2^6+2^5+2^0}_{\mbox{ by (\ref{eq3})}}\nonumber\\
   =\ & (2^8+2^8)+(2^7+2^6+2^5)+(2^7+2^6+2^5)+2^0\nonumber\\
   =\ & 2^9+2^8+2^7+2^6+2^0
\end{eqnarray}

Equation~\ref{eq4} shows that $11111^2$ has 10 bits. Next, when building Equation~\ref{eq5} we use Equation~\ref{eq4}.


\begin{eqnarray} \label{eq5}
    & (2^5+2^4+2^3+2^2+2^1+2^0)^2\nonumber\\
   =\ & (2^5)^2+2\cdot2^5(2^4+2^3+2^2+2^1+2^0)+\underline{(2^4+2^3+2^2+2^1+2^0)^2}\nonumber\\
   =\ & 2^{10}+2^6(2^4+2^3+2^2+2^1+2^0)+\underline{2^9+2^8+2^7+2^6+2^0}_{\mbox{ by (\ref{eq4})}}\nonumber\\
   =\ & (2^{10}+2^{10})+(2^9+2^8+2^7+2^6)+(2^9+2^8+2^7+2^6)+2^0\nonumber\\
   =\ & 2^{11}+2^{10}+2^9+2^8+2^7+2^0
\end{eqnarray}

Equation~\ref{eq5} shows that $111111^2$ has 12 bits.

\section{Formal Inductive Proof}

Based on Equations~\ref{eq0}--\ref{eq5}, it is reasonable to conjecture that for all $n=0,1,2,3,\ldots$,

\begin{equation} \label{eq6}
  (2^n+2^{n-1}+\cdots+2^0)^2=2^{2n+1}+2^{2n}+\cdots+2^{n+2}+2^0.
\end{equation}  
Note that on both sides of Equation~\ref{eq6} the powers of 2 are in descending order, so that the right hand side is instantiated by $2^0$ when $n=0$, by $2^3+2^0$ when $n=1$ and by $2^5+2^4+2^0$ when $n=2$, and so on.

Based on how we progress from Equation~\ref{eq0} to~\ref{eq5}, we have the following scheme of progression from the square of sum up to $n$-th power of 2, to the square of sum up to $(n+1)$-th power of 2.


\begin{eqnarray} \label{eq7} 
    & (2^{n+1}+2^n+\cdots+2^0)^2\nonumber\\
   =\ & (2^{n+1})^2+2\cdot2^{n+1}(2^n+2^{n-1}+\cdots+2^0)+\underline{(2^n+2^{n-1}+\cdots+2^0)^2}\nonumber\\
   =\ & 2^{2(n+1)}+2^{n+2}(2^n+2^{n-1}+\cdots+2^0)+\underline{2^{2n+1}+2^{2n}\cdots+2^{n+2}+2^0}_{\mbox{ by (\ref{eq6})}}\nonumber\\
   =\ & \left(2^{2(n+1)}+2^{2(n+1)}\right)+(2^{2n+1}+\cdots+2^{n+2})+(2^{2n+1}+\cdots+2^{n+2})+2^0\nonumber\\
   =\ & 2^{2(n+1)+1}+2^{2(n+1)}+\cdots+2^{(n+1)+2}+2^0
\end{eqnarray}

Based on the facts that Equation~\ref{eq0} holds, and that for all $n=0,1,2,3,\ldots$, if Equation~\ref{eq6} holds then Equation~\ref{eq7} holds, we can justly conclude that our conjecture that for all $n=0,1,2,3,\ldots$ Equation~\ref{eq6} holds, is true. This proves our Lemma. Then the Proposition follows. 

\end{document}
